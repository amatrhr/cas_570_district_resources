\documentclass{article}

\usepackage[
backend=biber,
style=alphabetic,
sorting=ynt
]{biblatex}

\addbibresource{thebib.bib}
\usepackage{graphicx} % Required for inserting images

\title{District Resources Dynamic Model Simulation}
\author{Charles Laurin \\CAS 570}
\date{September 2024}

\begin{document}
\maketitle


\section*{Introduction}
Public school districts in the United States are obligated to provide equal educational opportunities to all students; their finances are generally also subject to local tax funding under the management of elected school boards. This means that they operate under a relatively narrow range of student-to-resource ratios, and declining student populations are important predictors of school closure, a drastic adjustment to these ratios \cite{harris2023extreme}. 

This project asks how, when faced with uncertain enrollment trends and a need to maintain a proper assignment of resources to students, how might a school district make a stable, sustainable, and fair resource policy by maintaining a tolerable student-to-resource ratio, for example, maintaining a certain maximum student-to-teacher ratio or school size.

I will address this question by using a two-agent simulation (it might also be fair to call this a dynamical system model) of school openings and closings in a single school district, played out over 50 school years. One agent represents the population of public school students, which undergoes random demographic variation across years. The other agent represents the policy-making body of the district, which attempts to maintain a tolerable student-to-school ratio in the face of these demographic changes by predicting the change in enrollment at the end of each school year and opening or closing schools in response.

\section*{Methods}
\subsection*{Simplifying Assumptions}
I make several simplifying assumptions to justify using the ratio between the number of students in a district and the number of schools as a meaningful representation of the distribution of resources.
In the simulation, students are treated as indistinguishable in terms of need for resources and will not enter or leave school during the year; to minimize complexity, their decisions to enter or leave school between school years do not depend on the student-to-school ratio. School policy decisions are implemented instantaneously during the summer break between school years--a school can be built with no lead time. Every school is treated as having the same number of teachers.
\subsection*{Agent Behavior: District Population}
I will simulate uncertain patterns of student demographic change within an individual school district; for each Convex Growth/Decline
		


	
	Concave Growth/Decline
		
		

	Impulses 
		

	Flat with Noise

		
Set policy by learning EWMA parameters on first differences: Bayesian updating of an odd number (try n=5 or n=10) discrete ⍶ between 0 and 1–if the ratio is maintained, increase likelihood of choosing the ⍶ again–moving from a 20/100 chance of being chosen to a 24/100, while others now have a 19/100 chance.

	Exponentially weighted moving average: x{n+1}=x{n}+(1-)x{n-1}+(1-)2x{n-2}+... 

x{n+1}=x{n}+(1-)x{n-1}

Tolerances  for being over/under the ratio: -50\% to +50\% in 10\% increments. Example–if the district must have no more than 200 students per school, with 10\% tolerance, then any enrollment increase to a district with 2199 students in 100 schools  must result in building a new school. If this district has a negative tolerance of 50\% and  ever falls below 1000 students, it must close a school.


\subsection*{Agent Behavior: District Administration}
\cite{snyder1999understanding}
\cite{fogel1999inductive}
\subsection*{Independent Replicates}


Metrics
Mean Absolute % Error in student/resource ratio
For normalized time series:
MAD 	
RMSE

\medskip

\printbibliography
\end{document}
